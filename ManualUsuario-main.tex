\documentclass{GVT_CONAE_Class}

%%%%%%%%%%%%%%%%%%%%%%%%%%%%%%%%%%%%%%%%%%%%%%%%%%%%%%%%%%%%%%%%%%%%%%%%%%%%%%%%%%%%%%%%%%
%%% DATOS DEL DOCUMENTO A COMPLETAR
%%%%%%%%%%%%%%%%%%%%%%%%%%%%%%%%%%%%%%%%%%%%%%%%%%%%%%%%%%%%%%%%%%%%%%%%%%%%%%%%%%%%%%%%%%
% Nombre del proyecto
\Proyecto{Nombre del proyecto}
% Título del documento
\TituloDoc{Manual de usuario: Generación de metadatos}
% Código del documento
\CodigoDoc{GVT-AAA-PRO-DO-NUM}
% Versión del documento
\VersionDoc{v01}
% Autores y posición de cada uno
\AutorA{Ivanna Tropper}\PosicionAutorA{GVT - ACPP}
\AutorB{}\PosicionAutorB{}
\AutorC{}\PosicionAutorC{}
\AutorD{}\PosicionAutorD{}
% Revisores del documento, posición y fecha de revisión
\RevisorA{}\PosicionRevisorA{}\FechaRevisorA{}
\RevisorB{}\PosicionRevisorB{}\FechaRevisorB{}
\RevisorC{}\PosicionRevisorC{}\FechaRevisorC{}
\RevisorD{}\PosicionRevisorD{}\FechaRevisorD{}
% Aprobadores del documento, posición y fecha de aprobación
\AprobadorA{}\PosicionAprobadorA{}\FechaAprobadorA{}
\AprobadorB{}\PosicionAprobadorB{}\FechaAprobadorB{}
\AprobadorC{}\PosicionAprobadorC{}\FechaAprobadorC{}
\AprobadorD{}\PosicionAprobadorD{}\FechaAprobadorD{}
% Tipo de documento
\TipoDocumento{Manual de Usuario}
% Fecha de creación del documento
\FechaDoc{ColoqueFecha}
% Cambios del documento
\VersionA{}\FechaA{}\CambioA{}
\VersionB{}\FechaB{}\CambioB{}
\VersionC{}\FechaC{}\CambioC{}
\VersionD{}\FechaD{}\CambioD{}
\VersionE{}\FechaE{}\CambioE{}
\VersionF{}\FechaF{}\CambioF{}
\VersionG{}\FechaG{}\CambioG{}
% Documentos de referencia
\CodReferenciaA{}\TituloReferenciaA{}
\CodReferenciaB{}\TituloReferenciaB{}
\CodReferenciaC{}\TituloReferenciaC{}
\CodReferenciaD{}\TituloReferenciaD{}
\CodReferenciaE{}\TituloReferenciaE{}
\CodReferenciaF{}\TituloReferenciaF{}
% Documentos aplicables
\CodAplicableA{}\TituloAplicableA{Procedimiento de generación de metadatos de productos}
\CodAplicableB{}\TituloAplicableB{}
\CodAplicableC{}\TituloAplicableC{}
\CodAplicableD{}\TituloAplicableD{}
\CodAplicableE{}\TituloAplicableE{}
\CodAplicableF{}\TituloAplicableF{}
%%%%%%%%%%%%%%%%%%%%%%%%%%%%%%%%%%%%%%%%%%%%%%%%%%%%%%%%%%%%%%%%%%%%%%%%%%%%%%%%%%%%%%%%%%
%%%%%%%%%%%%%%%%%%%%%%%%%%%%%%%%%%%%%%%%%%%%%%%%%%%%%%%%%%%%%%%%%%%%%%%%%%%%%%%%%%%%%%%%%%
\begin{document}

%%%%%%%%%%%%%%%%%%%%%%%%%%%%%%%%%%%%%%%%%%%%%%%%%%%%%%%%%%%%%%%%%%%%%%%%%%%%%%%%%%%%%%%%%%
% Secciones obligatorias en todos los documentos
%%%%%%%%%%%%%%%%%%%%%%%%%%%%%%%%%%%%%%%%%%%%%%%%%%%%%%%%%%%%%%%%%%%%%%%%%%%%%%%%%%%%%%%%%%
%----------------------------------------------------------------------------
\section{Sobre el documento}
%----------------------------------------------------------------------------

    \subsection{Objetivos}
    
    \hl{Cambiar la el cls porque es viejo}
    El presente documento tiene por objetivo definir que son los metadatos en forma general y particularmente cómo están comuestos los perfiles de metadatos vigentes en la \gls{conae}.

    \subsection{Alcance}
    Está destinado a todo agente de la \gls{gvt} que deba utilizar metadatos, ya sea porque tiene qyue generarlos o porque los utilizará como parte de los productos distribuidos de la \gls{conae}.
    
    
    % Subsección de acrónimos
    \printglossary[type=\acronymtype, numberedsection, title=Lista de acrónimos y abreviaturas]
%%%%%%%%%%%%%%%%%%%%%%%%%%%%%%%%%%%%%%%%%%%%%%%%%%%%%%%%%%%%%%%%%%%%%%%%%%%%%%%%%%%%%%%%%%
% Fin de secciones obligatorias
%%%%%%%%%%%%%%%%%%%%%%%%%%%%%%%%%%%%%%%%%%%%%%%%%%%%%%%%%%%%%%%%%%%%%%%%%%%%%%%%%%%%%%%%%%
%----------------------------------------------------------------------------
\section{Introducción}
%----------------------------------------------------------------------------
Qué son los metadatos? Por qué son necesarios? 

%----------------------------------------------------------------------------
\section{Descripción de metadatos obligatorios}
%----------------------------------------------------------------------------
Los campos que componen los metadatos están clasificados en obligatorios y optativos. A continuación se describen los campos obligatorios.

\subsection{ID único}
Etiqueta:  <!--mdb:mi\_MI\_MD\_I-c--> 
Valor ejemplo : nombre del archivo, sin extensión
\begin{tabular}{c|c}
    Elemento & Código de identificación único del producto \\
    Definición & \\
    Obligatoriedad (tipo de elemento) & \\
    Multiplicidad & \\
    Estructura XML (espacio nombre) & \\
    Tipo de Dato & \\
    Dominio & \\
    Comentarios & \\
    Ejemplo & \\
\end{tabular}


\subsection{Idioma (fijo)}
Etiqueta: <!--mdb:dl\_lan-cod-->
Valor fijo: "SPA"
\begin{tabular}{c|c}
    Elemento & Código de identificación único del producto \\
    Definición & \\
    Obligatoriedad (tipo de elemento) & \\
    Multiplicidad & \\
    Estructura XML (espacio nombre) & \\
    Tipo de Dato & \\
    Dominio & \\
    Comentarios & \\
    Ejemplo & \\
\end{tabular}


\subsection{Codificación (fijo)}
Etiqueta <!--mdb:dl\_lan-cha-->
Valor fijo: "utf8"
\begin{tabular}{c|c}
    Elemento & Código de identificación único del producto \\
    Definición & \\
    Obligatoriedad (tipo de elemento) & \\
    Multiplicidad & \\
    Estructura XML (espacio nombre) & \\
    Tipo de Dato & \\
    Dominio & \\
    Comentarios & \\
    Ejemplo & \\
\end{tabular}


\subsection{Nivel jerárquico}
Tipo de dato
Etiqueta <!--mdb:ms\_MD\_MS--> línea 28
Codelist: MD\_ScopeCode
Valor: MD\_ScopeCode\_dataset
\begin{tabular}{c|c}
    Elemento & Código de identificación único del producto \\
    Definición & \\
    Obligatoriedad (tipo de elemento) & \\
    Multiplicidad & \\
    Estructura XML (espacio nombre) & \\
    Tipo de Dato & \\
    Dominio & \\
    Comentarios & \\
    Ejemplo & \\
\end{tabular}


\subsection{Nombre del nivel jerárquico} 
\subsubsection{Título de la serie}
Etiqueta: <!--mdb:ii\_MD\_DI-ci\_c-s-CI\_s-n--> línea 139
Texto libre
Valor:  Serie de productos derivados - Producto diario de la columna troposférica de NO2, TROPOMI/Sentinel-5p
\begin{tabular}{c|c}
    Elemento & Código de identificación único del producto \\
    Definición & \\
    Obligatoriedad (tipo de elemento) & \\
    Multiplicidad & \\
    Estructura XML (espacio nombre) & \\
    Tipo de Dato & \\
    Dominio & \\
    Comentarios & \\
    Ejemplo & \\
\end{tabular}


\subsubsection{identificador corto del nivel jerárquico}
Etiqueta: <!--mdb:ii\_MD\_DI-ci\_c-s-CI\_s-ii--> línea 142
Texto libre
Valor: TROPOMI-NO2diario
\begin{tabular}{c|c}
    Elemento & Código de identificación único del producto \\
    Definición & \\
    Obligatoriedad (tipo de elemento) & \\
    Multiplicidad & \\
    Estructura XML (espacio nombre) & \\
    Tipo de Dato & \\
    Dominio & \\
    Comentarios & \\
    Ejemplo & \\
\end{tabular}


\subsection{Punto de contacto del metadato  (fijo)}
\subsubsection{Datos del publicador}
Etiqueta
Valor
\subsubsection{Tipo de punto de contacto (fijo)}
Codelist: CI\_RoleCode
Etiqueta <!--mdb:ii\_MD\_DI-pc-CI\_R-r-1-->
Valor: publisher

\subsubsection{Nombre de la organización}
Texto libre
Etiqueta: <!--mdb:ii\_MD\_DI-poc-p-CI\_O-n1-->
Valor: CONAE - Unidad de Sistemas de Información

\subsubsection{Teléfono}
Texto libre
Etiqueta: <!--mdb:ii\_MD\_DI-poc-p-CI\_O-ci-CI\_C-p1-->
Valor: "+541143310074 ext 5311"

\subsubsection{Tipo de teléfono}
Codelist: CI\_telephoneTypeCode
Etiqueta:<!--mdb:ii\_MD\_DI-poc-p-CI\_O-ci-CI\_C-tp1-->
Valor: voice
\subsubsection{Dirección postal}
Texto libre
Etiqueta: <!--mdb:ii\_MD\_DI-poc-p-CI\_O-ci-CI\_C-a-CI\_A-dp1-->
Valor: Av. Paseo Colón 751

\subsubsection{Ciudad}
Texto libre
Etiqueta: <!--mdb:ii\_MD\_DI-poc-p-CI\_O-ci-CI\_C-a-CI\_A-c1-->
Valor: CABA

\subsubsection{Código postal}
Texto libre
Etiqueta: <!--mdb:ii\_MD\_DI-poc-p-CI\_O-ci-CI\_C-a-CI\_A-cp1-->
Valor: 1063

\subsubsection{País}
Texto libre
Etiqueta: <!--mdb:ii\_MD\_DI-poc-p-CI\_O-ci-CI\_C-a-CI\_A-p1-->
Valor: Argentina

\subsubsection{Dirección electrónica}
Texto libre
Etiqueta: <!--mdb:ii\_MD\_DI-poc-p-CI\_O-ci-CI\_C-a-CI\_A-em1-->
Valor: fgarciaferreyra@conae.gov.ar / emergencias@conae.gov.ar 

\subsubsection{Nombre individual}
Texto libre
Etiqueta: <!--mdb:ii\_MD\_DI-poc-p-CI\_O-i-CI\_I-n1-->
Valor: Unidad de Emergencias y Alerta Temprana

\subsubsection{Tipo puesto}
Texto libre
Etiqueta: <!--mdb:ii\_MD\_DI-poc-p-CI\_O-i-CI\_I-p1-->
Valor: Técnico

\begin{tabular}{c|c}
    Elemento & Código de identificación único del producto \\
    Definición & \\
    Obligatoriedad (tipo de elemento) & \\
    Multiplicidad & \\
    Estructura XML (espacio nombre) & \\
    Tipo de Dato & \\
    Dominio & \\
    Comentarios & \\
    Ejemplo & \\
\end{tabular}


\subsection{Fecha de los metadatos}
Etiqueta: <!--mdb:di\_MD\_D-df-MD\_F-fsc-CI\_C-d-CI\_D-d-->
Valor: fecha de Publicación del metadato 
(fecha de publicación de conae en el catálogo)

\subsection{Nombre del perfil de metadato(fijo)}
Etiqueta: <!--mdb:ms\_CI\_C-t-->	
Valor fijo:  Perfil metadato CONAE - ISO 19115-3

Etiqueta: <!--mdb:ms\_CI\_C-at-->	
Valor fijo: Metadatos raster - 2018 - V1

\subsection{Título}
Etiqueta: <!--mdb:ii\_MD\_DI-ci\_c-t-->
Valor ejemplo: es igual al atributo 1, definido por la IDE.

\subsection{Resumen}
Texto Libre
Etiqueta : <!--mdb:ii\_MD\_DI-a--> línea 152
Valor: El producto diario de NO2 presenta la densidad de la columna troposférica de dióxido de nitrógeno, derivada del producto NO2 L2 del sensor TROPOMI/Sentinel-5p (ESA) http://www.tropomi.eu. Se presenta en unidades de mol/m2 y con geometría WGS84 latitud/longitud, en formato GeoTIFF y para centros urbanos argentinos de interés.
El producto de promedio semanal de NO2 presenta la densidad de la columna troposférica de dióxido de nitrógeno, derivada del producto NO2 L2 del sensor TROPOMI/Sentinel-5p (ESA) http://www.tropomi.eu. Se presenta en unidades de mol/m2 y con geometría WGS84 latitud/longitud, en formato GeoTIFF y para centros urbanos argentinos de interés. 
El producto de desvío estándar semanal de NO2 presenta la densidad de la columna troposférica de dióxido de nitrógeno, derivada del producto NO2 L2 del sensor TROPOMI/Sentinel-5p (ESA) http://www.tropomi.eu. Se presenta en unidades de mol/m2 y con geometría WGS84 latitud/longitud, en formato GeoTIFF y para centros urbanos argentinos de interés.
El producto de promedio mensual de NO2 presenta la densidad de la columna troposférica de dióxido de nitrógeno, derivada del producto NO2 L2 del sensor TROPOMI/Sentinel-5p (ESA) http://www.tropomi.eu. Se presenta en unidades de mol/m2 y con geometría WGS84 latitud/longitud, en formato GeoTIFF y para centros urbanos argentinos de interés.
El producto de desvío estándar mensual de NO2 presenta la densidad de la columna troposférica de dióxido de nitrógeno, derivada del producto NO2 L2 del sensor TROPOMI/Sentinel-5p (ESA) http://www.tropomi.eu. Se presenta en unidades de mol/m2 y con geometría WGS84 latitud/longitud, en formato GeoTIFF y para centros urbanos argentinos de interés.

\subsubsection{Crédito}
Texto Libre
Etiqueta: <!--mdb:ii\_MD\_DI-c--> línea 156
Valor: CONAE - Unidad de Emergencias y Alerta Temprana, Subgerencia de Servicios al usuario, Gerencia de Vinculación Tecnológica. 

\subsubsection{Status}
Etiqueta: <!--mdb:ii\_MD\_DI-s--> línea 160
Valor: onGoing
Codelist: MD\_ProgressCode

\subsection{Fecha de creación del producto}
Formato fecha (formato especificado en el perfil)
Etiqueta : <!--mdb:ii\_MD\_DI-ci\_c-d--> línea 129
Valor: 2018-07-15T18:33:27

\subsection{Frecuencia de mantenimiento}
Etiqueta: <!--mdb:ii\_MD\_DI-rm-MD\_MI-f-->
valor: AsNeeded
Codelist: MD\_MaintenanceFrequencyCode

\subsection{Tema}
Etiqueta: <!--mdb:ii\_MD\_DI-tc--> línea 242
Valor: environment
Codelist: MD\_TopicCategoryCode

\subsection{Palabras claves}
\subsubsection{Tema}
Texto libre
Etiqueta: <!--mdb:ii\_MD\_DI-MD\_Kw-tema-->  línea 296
Valor ejemplo: Dióxido de nitrógeno

\subsubsection{Tema1}
Etiqueta: <!--mdb:ii\_MD\_DI-MD\_Kw-tema1--> línea 307
Valor ejemplo: Calidad del Aire

\subsubsection{Lugar}
Etiqueta:<!--mdb:ii\_MD\_DI-MD\_Kw-place--> línea 330
Valor ejemplo: Ciudad X según nomenclatura de BAHRA http://www.bahra.gob.ar/

\subsubsection{Plataforma}
Etiqueta:<!--mdb:ii\_MD\_DI-MD\_Kw-plat-->  línea 341
Valor ejemplo: Sentinel-5p

\subsubsection{Temporal}
Etiqueta:<!--mdb:ii\_MD\_DI-MD\_Kw-temp--> línea 318
Valor ejemplo: a) actualización diaria, b) y c) actualización semanal, d) y e) actualización mensual

(se pueden agregar tantas palabras claves se desea, siempre y cuando se respete la estructura de las etiquetas para especificar tipo)
repetir esta estructura:
<mri:descriptiveKeywords>
	<mri:MD\_Keywords>
		<mri:keyword>
			<gco:CharacterString><!--mdb:ii\_MD\_DI-MD\_Kw-tema--></gco:CharacterString>
		</mri:keyword>
		<mri:type>
			<mri:MD\_KeywordTypeCode codeListValue="theme" codeList="http://standards.iso.org/ittf/PubliclyAvailableStandards/ISO\_19139\_Schemas/resources/codelist/ML\_gmxCodelists.xml#MD\_KeywordTypeCode"/>
		</mri:type>
	</mri:MD\_Keywords>
</mri:descriptiveKeywords>

\subsection{Restricciones}
\subsubsection{Legal}
Etiqueta: <!--mdb:ii\_MD\_LC-ac-->
Valor fijo: copyright

\subsubsection{Uso}
Etiqueta: <!--mdb:ii\_MD\_LC-uc-->
Valor fijo: licenceUnrestricted

\subsubsection{Otras restricciones o forma de citar el producto}
Etiqueta: <!--mdb:ii\_MD\_LC-oc--> línea 360
Texto libre
Valor ejemplo: La descarga y/o uso de cualquiera de estos productos SAOCOM de Nivel 2 y Superior SAOCOM implica por consiguiente la aceptación de los presentes Términos y Condiciones de Uso y el reconocimiento y respeto de los derechos de Propiedad Intelectual y de Derecho de Autor de los Productos. Se deberá indicar la siguiente leyenda “Producto SAOCOM® - ©CONAE - año de adquisición. Todos los derechos reservados” en todas las publicaciones, resultados, productos derivados y demás usos que los usuarios les den a dichos Productos.

\subsection{Tipo de representación espacial}
Etiqueta: <!--mdb:ii\_MD\_DI-srt--> línea 221
Codelis: MD\_SpatialRepresentationTypeCode
Valor fijo: GRID

\subsection{Punto de contacto del creador del recurso}
Etiqueta: <!--mdb:c\_CI\_R-r-->
Codelist: CI\_RoleCode
Valor fijo: pointOfContact

Etiqueta: <!--mdb:c\_CI\_R-p-CI\_O-n-->
Texto libre
Valor fijo: CONAE - Atencion al Usuario

etiqueta: <!--mdb:c\_CI\_R-p-CI\_O-ci-CI\_C-a-CI\_A-dp-->
Texto libre
Valor fijo (direccion postal) : Paseo Colon 751

etiqueta: <!--mdb:c\_CI\_R-p-CI\_O-ci-CI\_C-a-CI\_A-c-->
Texto libre
Valor fijo (ciudad): CABA

etiqueta:<!--mdb:c\_CI\_R-p-CI\_O-ci-CI\_C-a-CI\_A-pc-->
Texto libre
Valor fijo (codigo postal): C1063ACH

etiqueta: <!--mdb:c\_CI\_R-p-CI\_O-ci-CI\_C-a-CI\_A-p-->
Texto libre
Valor fijo (País): Argentina

etiqueta: <!--mdb:c\_CI\_R-p-CI\_O-ci-CI\_C-a-CI\_A-em-->
Texto libre
Valor fijo (correo): atencion.usuario@conae.gov.ar

Si se quiere agregar otro punto de contacto hay que repetir la estructura del etiquetas. 

\subsection{Escala espacial del dato}
\subsubsection{Escala (Resolución espacial)}
Etiqueta: <!--mdb:ii\_MD\_DI-sr-es-MD\_RF-d--> línea 227
Tipo de dato: Integer
Valor ejemplo: 250000

\subsubsection{Escala de trabajo}
Etiqueta: <!--mdb:ii\_MD\_DI-sr-lod--> línea 236
Texto libre
Valor ejemplo: Escala de trabajo aproximada 1:250000, lo que equivale a una resolución de pixel de 62.5m.

Fórmula para el cálculo de la escala de trabajo y tamaño de píxel
$Tamaño de celda = Escala * 0.0254 / 96
Escala = Tamaño de celda * 96 / 0.0254
Escala = (7x3,5 km2) * 96 / 0.0254 = 92598,42519685 => 1:100.000 ?$

\subsection{Extensión geográfica}
\subsubsection{Coordenada extrema oeste}
Etiqueta: <!--mdb:ii\_MD\_DI-e-E\_EX-ge-WBL--> línea 249
Decimales
Valor : UL

\subsubsection{Coordenada extrema este}
Etiqueta: <!--mdb:ii\_MD\_DI-e-E\_EX-ge-EBL--> línea 252
Decimales
Valor: UR

\subsubsection{Coordenada extrema sur}
Etiqueta: <!--mdb:ii\_MD\_DI-e-E\_EX-ge-SBL--> línea 255
Decimales
Valor: LR

\subsubsection{Coordenada extrema norte}
Etiqueta: <!--mdb:ii\_MD\_DI-e-E\_EX-ge-NBL--> línea 258
Decimales
Valor: LL

\subsection{Sistema de referencia (fijo)}
Etiqueta: <!--mdb:rsi\_MD\_RS-md\_i-t--> 
Valor ejemplo: Sistema Geográfico Mundial

Etiqueta: <!--mdb:rsi\_MD\_RS-md\_i-c-->  
Valor ejemplo: EPSG:4326(WGS84)

\subsection{Formato de distribución del recurso(fijo)}
Etiqueta: <!--mdb:di\_MD\_D-df-MD\_F-fsc-CI\_C-t-->  
Valor ejemplo: Geotiff

\subsection{Nombre del enlace}
Etiqueta: <!--mdb:di\_MD\_D-to-MD\_DTO-ol-CI\_OR-n-->
Valor fijo: Recurso para Descargar




%----------------------------------------------------------------------------
\section{Instalación} % Si corresponde
%----------------------------------------------------------------------------



%----------------------------------------------------------------------------
\section{Descripción de las herramientas paso a paso}
%----------------------------------------------------------------------------





%*************************************************************************************
%% Referencias -- Si no hay referencias, comentar las siguientes líneas
%*************************************************************************************
\addcontentsline{toc}{section}{Referencias} % Agrego la sección referencias al índice
\bibliographystyle{ieeetr}  % Formato de las referencias: IEEE
\bibliography{referencias}  % Archivo con la bibliografía


%*************************************************************************************
% A partir de la siguiente línea todas las secciones que se coloquen serán anexos
%*************************************************************************************
\newpage % Salto de página
\appendix
%********************


\section{Descripción de los campos presentes en los metadatos} \label{AP:descrcampos}

\begin{landscape}
 \begin{longtable}[c]{| p{1.5cm} | p{4cm} | p{6cm} | p{4cm}| p{6cm} |}
 \caption{Long table caption.\label{long}}\\

 \hline
 \multicolumn{5}{| c |}{Begin of Table}\\
 \hline
 L\'inea de referencia & Nombre del campo & Etiqueta & Definici\'on & Ejemplo\\
 \hline
 \endfirsthead

 \hline
 \multicolumn{5}{|c|}{Continuation of Table \ref{long}}\\
 \hline
 L\'inea de referencia & Nombre del campo & Etiqueta & Definici\'on & Ejemplo\\
 \hline
 \endhead

 \hline
 \endfoot

 \hline
 \multicolumn{5}{| c |}{End of Table}\\
 \hline\hline
 \endlastfoot
 
  6 & ID unico (1) & <!--mdb:mi\_MI\_MD\_I-c--> & Referencia inequívoca de un recurso en contexto dado & 630ef5c7-8301-4805-9602-4e93a3c5516e \\
 28 & Nivel jerárquico (4) & <!--mdb:ms\_MD\_MS--> & Tipo de dato & Codelist: MD\_ScopeCode \hl{sin par en tabla} \\
 35 & Punto de contacto del creador del recurso (17.1) \hl{Tipo} & <!--mdb:c\_CI\_R-r--> & Identificación y manera de comunicarse con, persona y organización asociada con la generación del recurso & \hl{agregar}\\
% % %    40 & Nombre de la organización (17.2) & <!--mdb:c\_CI_R-p-CI\_O-n--> & \\
% % %    47 & Dirección postal (17.3) & <!--mdb:c_CI_R-p-CI_O-ci-CI_C-a-CI_A-dp--> & \\
% % %    50 & Ciudad (17.4) & <!--mdb:c_CI_R-p-CI_O-ci-CI_C-a-CI_A-c--> & \\
% % %    53 & Código postal (17.5) & <!--mdb:c_CI_R-p-CI_O-ci-CI_C-a-CI_A-pc--> & \\
% % %    56 & País (17.5) & <!--mdb:c_CI_R-p-CI_O-ci-CI_C-a-CI_A-p--> & \\
% % %    59 & Email (17.6)& <!--mdb:c_CI_R-p-CI_O-ci-CI_C-a-CI_A-em--> & \\
% % %    72 & Fecha de creación del metadato (No está)& <!--mdb:di\_CI\_D-d-dt-c-->& \\
124 & Título (9) & <!--mdb:ii\_MD\_DI-ci\_c-t--> & Nombre por el que se conoce formalmente el recurso (capa o archivo digital), asignado por el autor u organismo responsable (creadores) & \hl{agregar}\\
129 & Fecha de referencia (11) & <!--mdb:ii\_MD\_DI-ci\_c-d--> & Fecha de referencia del recurso & \hl{agregar}\\
132 & Tipo de fecha de referencia & \hl{agregar} & Tipo de fecha de referencia del recurso & Codelist: CI\_DateTypeCode\\

139 & Nombre del nivel jerárquico (5.1) & <!--mdb:ii\_MD\_DI-ci\_c-s-CI\_s-n--> & Título de la serie & Serie de productos derivados - Índice de Vegetación obtenido con Radar \hl{sin par en tabla}\\
142 & Identificador corto del niver jerárquino (5.2) & <!--mdb:ii\_MD\_DI-ci\_c-s-CI\_s-ii--> & Título corto de la serie & SAR-IRV \hl{sin par en tabla}\\

% % %    145 & Ubicación del catálogo de la serie (no está) & <!--mdb:ii\_MD\_DI-ci\_c-s-CI\_s-p--> & \\
152 & Resumen (10.1) & <!--mdb:ii\_MD\_DI-a--> & Relato sintético del contenido del recurso, complementario a la DESCRIPCIÓN. El mismo nos permite una revisión rápida del recurso asociado al metadato & \hl{texto libre}\\

156 & Crédito (10.2) & <!--mdb:ii\_MD\_DI-c--> & & \hl{sin par en tabla} \\
160 & Status (10.3) & !--mdb:ii\_MD\_DI-s--> & & \hl{sin par en tabla} \\
% % %    170 & Datos del Publicador (DP)Nombre de la organización \hl{tipo} & <!--mdb:ii_MD_DI-poc-p-CI_O-n1--> & \\
% % %    177 & DP telefono & <!--mdb:ii_MD_DI-poc-p-CI_O-ci-CI_C-p1-->
% % %    187 & DP Dirección postal & <!--mdb:ii_MD_DI-poc-p-CI_O-ci-CI_C-a-CI_A-dp1-->
% % %    190 & DP Ciudad & <!--mdb:ii_MD_DI-poc-p-CI_O-ci-CI_C-a-CI_A-c1-->
% % %    193 & DP Código postal & <!--mdb:ii_MD_DI-poc-p-CI_O-ci-CI_C-a-CI_A-cp1-->
% % %    196 & DP País & <!--mdb:ii_MD_DI-poc-p-CI_O-ci-CI_C-a-CI_A-p1-->
% % %    199 & DP Email & <!--mdb:ii_MD_DI-poc-p-CI_O-ci-CI_C-a-CI_A-em1-->
% % %    208 & DP Nombre individual & <!--mdb:ii_MD_DI-poc-p-CI_O-i-CI_I-n1-->
% % %    211 & DP Tipo puesto & <!--mdb:ii_MD_DI-poc-p-CI_O-i-CI_I-p1-->


221 & Tipo de representación espacial (16) & <!--mdb:ii\_MD\_DI-srt--> & & grid \hl{sin par en tabla}\\
228 & Escala (Resolución espacial) (18.1) & <!--mdb:ii\_MD\_DI-sr-es-MD\_RF-d--> & \\
237 & Escala de trabajo (18.2) & <!--mdb:ii\_MD\_DI-sr-lod--> & \\
242 & Tema (13) & <!--mdb:ii\_MD\_DI-tc--> & \\
250 & Extensión geográfica: UL (19.1) & <!--mdb:ii\_MD\_DI-e-E\_EX-ge-WBL--> & \\
253 & Extensión geográfica: UR (19.2) & <!--mdb:ii\_MD\_DI-e-E\_EX-ge-EBL--> & \\
256 & Extensión geográfica: LR (19.3) & <!--mdb:ii\_MD\_DI-e-E\_EX-ge-SBL--> & \\
259 & Extensión geográfica: LL (19.4) & <!--mdb:ii\_MD\_DI-e-E\_EX-ge-NBL--> & \\
268 & Frecuencia de mantenimiento (12) & <!--mdb:ii\_MD\_DI-rm-MD\_MI-f--> & \\
% % %    276 & Quicklook & <!--mdb:ii\_MD\_DI-go-MD\_BG-fn-a--> & \\
% % %    286 & Quicklook 2 & !--mdb:ii\_MD\_DI-go-MD\_BG-fn-b--> & \\
296 & Tema (14.1) & <!--mdb:ii\_MD\_DI-MD\_Kw-tema--> & \\
307 & Tema 1 (14.2) & <!--mdb:ii\_MD\_DI-MD\_Kw-tema1--> & \\
318 & Temporal (14.5) & <!--mdb:ii\_MD\_DI-MD\_Kw-temp--> & \\
330 & Lugar (14.3) & <!--mdb:ii\_MD\_DI-MD\_Kw-place--> & \\
341 & Plataforma (14.4) & <!--mdb:ii\_MD\_DI-MD\_Kw-plat--> & \\
360 & Otras restricciones o forma de citar el producto (15.3) & <!--mdb:ii\_MD\_LC-oc--> & \\
% % %    391 & Link del catálogo & <!--mdb:ci\_MD\_FCD-fcc-CI\_C-or-CI\_OR-l--> & \\
% % %    394 & Protocolo del link del catálogo & <!--mdb:ci\_MD\_FCD-fcc-CI\_C-or-CI\_OR-p--> & \\
% % %    397 & Nombre de tipo de recurso & <!--mdb:ci\_MD\_FCD-fcc-CI\_C-or-CI\_OR-n--> & \\
% % %    400 & Descripción del tipo de catálogo & <!--mdb:ci\_MD\_FCD-fcc-CI\_C-or-CI\_OR-d--> & \\
425 & Fecha de los metadatos (7) & <!--mdb:di\_MD\_D-df-MD\_F-fsc-CI\_C-d-CI\_D-d--> & \\
% % %    442 & Link de descarga & <!--mdb:di\_MD\_D-to-MD\_DTO-ol-CI\_OR-l--> & \\
% % %    446 & Protocolo de descarga & <!--mdb:di\_MD\_D-to-MD\_DTO-ol-CI\_OR-p--> & \\
% % %    449 & Tipo de descarga & <!--mdb:di_MD_D-to-MD_DTO-ol-CI_OR-n--> & \\
% % %    453 & Descripción de la descarga & <!--mdb:di\_MD\_D-to-MD\_DTO-ol-CI\_OR-d--> & \\
 
\end{longtable}
\end{landscape}

\end{document}
